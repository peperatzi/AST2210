\documentclass{emulateapj}
%\documentclass[12pt,preprint]{aastex}

\usepackage{graphicx}
\usepackage{float}
\usepackage{amsmath}
\usepackage{amsfonts}
\usepackage{epsfig,floatflt}
\usepackage{braket}
%\usepackage{gensymb}

\begin{document}

\title{Determining the power spectrum of the COBE DMR sky maps: Analysis of four-year data}

\author{Hans-Petter Harveg}

\email{hanspeph@student.matnat.uio.no}

\altaffiltext{1}{Institute of Theoretical Astrophysics, University of
  Oslo, P.O.\ Box 1029 Blindern, N-0315 Oslo, Norway}

%\date{Received - / Accepted -}

\begin{abstract}
  We apply Marcov Chain Monte Carlo data analysis to the four-year COBE-DMR observations were we focus on two important cosmological parameters; $Q$, the \textit{amplitude} and $n$, the \textit{tilt} of the CMB spectrum.
\end{abstract}
\keywords{cosmic microwave background --- cosmology: observations --- methods: statistical}

\section{Introduction}
\label{sec:introduction}

\subsection{The Cosmic Microwave Background (CMB)}
\label{subsec:cmb_description}
  The CMB is a collection of the oldest phothons in the universe. These photons were released about $380 \ 000$ years after the Big Bang, when the temperature fell below $3000K$. Before that time, all space was filled with a plasma of free electrons and free photons and the photons were trapped whithin this plasma, because they frequently got scattered around in the plasma. Once the temperature fell below $3000K$, suddenly all free electrons effectively "disappeared", merging with protons into the hydrogen. The trapped photons were then released and free to move througout the universe.

  Because of the expansion of the universe, the wavelength of these released photons are streched. This caused the temperature of the photons to decrease by about $1100K$, from about $3000K$ to about$2.7K$, which is the wavelength that corresponds to microwaves and is today observable at all radio frequencies. Essensially, the CMB map shows the matter density shortly after the Big Bang. In other words; the hotter the photons, the higher the denisty.

\subsection{Spherical harmonic decomposition}

  In our analysis, we therefore need to find the variation in CMB temperature as a function of the position on the sky. To do so, we will use a variation of Fourier transforms, namely spherical harmonic decomposition which is just a spherical equivalent to Fourier transforms. Following the (Eriksen and Ruud, 2017)\ref{bib:assignment}, we have

\begin{equation}
\Delta T(\hat{n}) = \Sigma_{\ell=0}^{\ell_{max}}\Sigma_{m=-\ell}^\ell a_{\ell m}Y_{\ell m}(\hat{n}),\\[5pt]
\end{equation}
where $a_{\ell m}$ is the amplitudes of the waves, $\ell$ and $m$ are two charecteristic numbers, describing each wave mode. $\ell$ describe the effective wavelength and $m$ describe the phase.

  By applying sperical harmonic depomposition, we will use the \textit{amplitude} of the signal as a function of wavelength to produce what is called the \textit{angular power spectrum}. By comparing our predictions to the measured data, we can find a best-fit of the real picture of the universe.

  From now on, we will use bold upper case letters to denote matrices, bold lower case letters to denote vectors, and use \textit{p} as a \textit{pixel label}.

\section{Method}
\label{sec:method}

\subsection{Data model}
\label{subsec:data_model}
 
We will now describe the methods used in this Letter. Following (Eriksen and Ruud, 2017), we model our data as a sum of a CMB component and an instrumental noise component,
\begin{equation}
d(\hat{n}) = s(\hat{n}) + n(\hat{n}) + f(\hat{n}),\\[10pt]
\label{eq:equation_1}
\end{equation}
where \textit{d} is the observed signal, \textit{s} is the true CMB signal, \textit{n} is the noise, \textit{f} is possible non-cosmoligical foreground signals, and $\hat{n}$ is the direction. We assume that $s$, $n$ and $f$ are non-correlating,  so all cross products will have zero mean. 

  The covariance matrix of \textit{d} is therefore given by

\begin{equation}
\textbf{C} \equiv \braket{\textbf{d}\textbf{d}^T} \equiv \textbf{S} + \textbf{N} + \textbf{F},\\[10pt]
\end{equation}
where \textbf{S} is the covariance matrix of the CMB signal, \textbf{N} is the covariance matrix of the noise, and \textbf{F} is the covariance matrix of the foregrounds. In our case, each of these matrices will be an $N_{pix}\times N_{pix}$-matrix.

  We assume that the noise is Gaussian-distributed and uncorrelated between pixels. The standard deviation is given by $\sigma_p$. The noise covariance matrix is diagonal with the usual variance located along the diagonal

\begin{equation}
N_{ij} = \braket{n_in_j} = \sigma_i^2\delta_{ij},\\[10pt]
\end{equation}
where $i$ and $j$ are two pixel indices.

  Next, we assume that \footnote{Eriksen and Ruud, 2017} the CMB field is Gaussian and isotropic, but this time correlated between pixels. In this case the signal covariance matrix is therefore

\begin{equation}
S_{ij} = \frac{1}{4\pi}\Sigma_{\ell=0}(2\ell + 1)(\textit{b}_\ell\textit{p}_\ell)^2 C_\ell P_\ell(\cos{\theta_{ij}}),\\[10pt]
\end{equation}
where $b_\ell$ and $p_{\ell}$ is the \textit{instrumental beam} and \textit{pixel window} respectively, both described in \ref{sec:introduction}. $P_\ell(\theta_{is})$ is the legender polynomials where $\theta_{is}$ is the angle between pixels two pixels $i$ and $j$.

  Following (Bond and Efstathiou, 1987), to build the power spectrum $C_\ell$ we use
\begin{equation}
C_\ell = \frac{4\pi}{5}Q^2\frac{\Gamma{(\ell + \frac{n - 1}{2})}\Gamma{(\frac{9-n}{2})}}{\Gamma{(\ell + \frac{5-n}{2})}\Gamma{(\frac{3+n}{2})}},\\[10pt]
\end{equation}
We remove the monopole and the diploe by setting $C_0 = C_1 = 0$. Factorials are computational heavy. We therefore rewrite the expression by taking advantage of its recursive nature 

  To remove the monopole and dipole, we add a final term
\begin{equation}
\textbf{F} = \lambda\textbf{f}\textbf{f}^T,\\[10pt]
\end{equation}
where $\lambda$ is large constant, \textbf{f} is the known structure on the sky one wants to be insensitive to.

\begin{equation}
\textbf{C} = \textbf{S} + \textbf{N} + \textbf{F},\\[10pt]
\end{equation}
where the induvidual matrices are described above.

\subsection{Likelihood analysis}
\label{subsec:likelihood_analysis}
  To find the best-fit values of $Q$ and $n$ we use the so-called \textit{maximum-likelihood} framework (Eriksen and Ruud, 2017) defined by 
\begin{equation}
\mathcal{L}(Q,n) = p(\textbf{d}Q,n).\\[10pt]
\end{equation}

The joint distribution of $Q$ and $n$ is given by a multi-variate Gaussian distribution, with \textbf{d} as its variable. Using the covariance matrix described above, we thus have

\begin{equation}
-2\log{\mathcal{L}(Q,n)} = \textbf{d}^TC^{-1}\textbf{d} + \log{|C|} + constant.\\[10pt]
\label{eq:log_likelihood}
\end{equation}
We will use (eq. \ref{el:log_likelihood}) evaluate different combinations of $Q$ and $n$. Because of numerical errors, we need to perform all our calculations in \textit{log-uniuts}, then exponentiate at the end.

  Finaly we produce a two-dimensional contour plot based on the likelihood of the combinations of $Q$ and $n$. Markov Chain Monte Carlo is applyed to solve the likelyhood.

  Finally we use the two-dimensional data to calculate induvidual distributions for $Q$ and $n$ by
\begin{equation}
\mathcal{L}(Q) = \int_{inf}^{inf} \mathcal{L}(Q, n) \ dn \ \ \text{and} \ \ \mathcal{L}(n) = \int_{inf}^{inf} \mathcal{L}(Q, n) \ dQ,\\[10pt]
\end{equation}
for $Q$ and $n$ respectively.

% Define the power law model in terms of $Q$ and $n$. 


\section{Data}
\label{sec:data}
  In our analysis we use the following data
\begin{itemize}
\item Two sets of data set from observations taken at 53GHz and 90GHz. Both files contains 1941 pixels. Each pixel has a linear size of $3.7^\circ \ (230')$. From a total of 3072, 1131 have been removed by a binary mask.
\item Two corresponding files for the variance (rms) of each pixel.
\item A null transparancy mask binary mask for "hide" undesired data
\item Instrumental beam data
\end{itemize}

All data used in this experiment can be found here.

% Summarize properties of data. Which data are used (experiment,
% frequencies etc.)? Pixel resolution ($N_{\textrm{side}}$),
% $\ell_{\textrm{max}}$ -- everything necessary to repeat the analysis
% for other researchers.

% Show a sky map of the smoothed data. Use the Healpix routine
% ``smoothing'' to do this; it works just like anafast. Smooth with a
% $7^{\circ}$ beam, and plot with ``map2gif''. Show the RMS pattern as
% well. 


\section{Results}
\label{sec:results}

% Show the 2D likelihood contours. Summarize constraints on $Q$ and
% $n$. 

%\begin{figure}[t]
%\mbox{\epsfig{figure=../pixel_53Hz_dist.png,width=\linewidth,clip=}}
%\caption{Pixel plot of the distribution of $n$ of $Q$ for the $53Ghz$ band.}
%\label{fig:figure_53_label}
%\end{figure}

%\begin{figure}[t]
%\mbox{\epsfig{figure=../pixel_90Ghz_dist.png,width=\linewidth,clip=}}
%\caption{Pixel plot of the distribution of $n$ of $Q$ for the $90GHz$ band.}
%\label{fig:figure_90_label}
%\end{figure}

%\begin{figure}[t]
%\mbox{\epsfig{figure=../prob_n_dist.png,width=\linewidth,clip=}}
%\caption{Plot of the induvidual probability density of $n$ for the both 50 and $90GHz$ band.}
%\label{fig:figure_n_dist_label}
%\end{figure}

%\begin{figure}[t]
%\mbox{\epsfig{figure=../prob_Q_dist.png,width=\linewidth,clip=}}
%\caption{Plot of the induvidual probability density of $Q$ for the both 50 and $90GHz$ band.}
%\label{fig:figure_Q_dist_label}
%\end{figure}

%\begin{deluxetable}{cccc}
%    %\tablewidth{0pt}
%    \tablecaption{\label{tab:jwst}}
%    \tablecomments{JWST resolution }
%    %\tablecolumns{4}
%    \tablehead{ a & b & c & d  } 
%        \startdata
%        1 & 2 & 3 & 4\\
%        4 & 5 & 6 & 7\\
%        7 & 8 & 9 & 10\\
%        \enddata
%\end{deluxetable}

\section{Conclusions}
\label{sec:conclusions}

% Summarize results. Discuss their importance, referring to the
% discovery to the initial seeds for structure formation. Mention that
% these results are in good agreement with expectations from
% inflationary theory.

The Marcov Chain Monte Carlo algorithm 



\begin{acknowledgements}
  We want to thank Hans-Kristian Eriksen and Ainar Drews for guiding us through this study. Thanks to Daniel Heinesen for giving us pointers on how to solve the max funtion in the Metropolis Hastings algorithm.
\end{acknowledgements}

\begin{thebibliography}{}

\bibitem[Project assignment AST2210 - Analysis of four-year COBE-DMR DATA]{eriksen:2017} H.K.Eriksen, T.M Ruud
\label{bib:assignment}

\bibitem[G{\'o}rski et al.(1994)]{gorski:1994} G{\'o}rski, K. M.,
  Hinshaw, G., Banday, A. J., Bennett, C. L., Wright, E. L., Kogut,
  A., Smoot, G. F., and Lubin, P.\ 1994, ApJL, 430, 89
\label{bib:gorsky}

\end{thebibliography}


\end{document}
