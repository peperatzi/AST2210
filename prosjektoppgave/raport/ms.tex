\documentclass{emulateapj}
%\documentclass[12pt,preprint]{aastex}

\usepackage{graphicx}
\usepackage{float}
\usepackage{amsmath}
\usepackage{epsfig,floatflt}



\begin{document}

\title{On determining the spectrum of primordial inhomogeneity from the COBE DMR Sky Maps:Results of two-year data analyses}

\author{Hans-Petter Harveg}

\email{ola.d.nordmann@astro.uio.no}

\altaffiltext{1}{Institute of Theoretical Astrophysics, University of
  Oslo, P.O.\ Box 1029 Blindern, N-0315 Oslo, Norway}


%\date{Received - / Accepted -}

\begin{abstract}
  Data analysis has been applied to the two-year COBE-DMR 53 and 90 GHz sky maps. The Bayesian power spectrum estimation results are consistent with the Harrison-Zel'dovich $n = 1$ model. The maximum likelihood estimates of of the usual parameters defining the power spectrum of primordial perturbations are $n = 1.22$ and $Q = 17 \ \mu K$ including (excluding) 
\end{abstract}
\keywords{cosmic microwave background --- cosmology: observations --- methods: statistical}

\section{Introduction}
\label{sec:introduction}
  The determination of the power spectrum of primordial inhomogeneities and their consiscency with the pridictions of inflation are critical issues in contemporary cosmology. Standard inflationary scenarios imply a power-law spectrum $P(k) \propto k^n$ with $n \approx 1$ on the scales probed in the COBE DMR sky maps. Previous attepts to detirmine the primordial power spectrum from the DMR found both relatively steep spectra and sensitivity to the inclusion or exclusion of the quadrupole in their results. Most methods have employed approximate statistical techuniques and have relied on Monte Carlo techniques to access and/or calibrate the final results.

  In this \textit{Letter} the COBE DRM two-year 53 and 90 GHz data are analizied to determine the primordial power spectrum using the methods described in Gòrski (1994). Spesifically, the sky maps are Fourier decomposed in the basis of orthonormal functions on the Galaxy-cut sky to yield a set of harmonic mode coefficients, which are linear in pixel temperatures. These are then used in a maximum likelihood analyses to infer the parameters of the theoretical ansisotropy models. The merits of the present method are (1) harmonic mode coupling is explicity accounted for by contruction of the orthonormal functions of the Galaxy-cut sky, (2) since the harmonic modes have a Gaussian probability distribution an \textit{exact} likelihood function for the model parameters can be employed, (3) the monopole and dipole components, which are physically irrelevant for the power spectrum estimation, are algegraicly excluded, and (4) the technique permits a simultaneous analysis of different frequency maps taking full advantage of both the auto- and cross-correlation information in the data.

  In this analysis we Fourier decompose the two-year DMR sky maps over the $\ell = 46$, where the DMR beam response has fallen to $\approx 0.2$, and the multipole amplitude is entirely noise dominated. No attempts is made to model and subtract formally the diffuse high-latitude Galactic emmisiom, which is predominantly quadrupolar in nature. Therefor, in what follows, the power spectrum parameters are derived for two cases: one in which data spanning the multipole range$\ell \in [2,46]$ are used, and the other in which the quadropole ($\ell = 2$) mode is excluded.

  Hereafter, bold upper case letters denote matrices; vectors are denoted by an arrow above the letter; and \textit{p} is a pixel label.

\section{Method}
\label{sec:method}


Describe method. Define data model and likelihood. Outline how the
likelihood was computed (grid or MCMC).

Define the power law model in terms of $Q$ and $n$. 

\section{Data}
\label{sec:data}

Summarize properties of data. Which data are used (experiment,
frequencies etc.)? Pixel resolution ($N_{\textrm{side}}$),
$\ell_{\textrm{max}}$ -- everything necessary to repeat the analysis
for other researchers.

Show a sky map of the smoothed data. Use the Healpix routine
``smoothing'' to do this; it works just like anafast. Smooth with a
$7^{\circ}$ beam, and plot with ``map2gif''. Show the RMS pattern as
well. 

\section{Results}
\label{sec:results}


Show the 2D likelihood contours. Summarize constraints on $Q$ and
$n$. 


\section{Conclusions}
\label{sec:conclusions}

Summarize results. Discuss their importance, referring to the
discovery to the initial seeds for structure formation. Mention that
these results are in good agreement with expectations from
inflationary theory.



%\begin{figure}[t]
%
%\mbox{\epsfig{figure=filename.eps,width=\linewidth,clip=}}
%
%\caption{Description of figure -- explain all elements, but do not
%draw conclusions here.}
%\label{fig:figure_label}
%\end{figure}



\begin{deluxetable}{lccc}
%\tablewidth{0pt}
\tablecaption{\label{tab:results}}
\tablecomments{Summary of main results.}
\tablecolumns{4}
\tablehead{Column 1  & Column 2 & Column 3 & Column 4}
\startdata
Item 1 & Item 2 & Item 3 & Item 4
\enddata
\end{deluxetable}



\begin{acknowledgements}
  Who do you want to thank for helping out with this project?
\end{acknowledgements}

\begin{thebibliography}{}

\bibitem[G{\'o}rski et al.(1994)]{gorski:1994} G{\'o}rski, K. M.,
  Hinshaw, G., Banday, A. J., Bennett, C. L., Wright, E. L., Kogut,
  A., Smoot, G. F., and Lubin, P.\ 1994, ApJL, 430, 89

\end{thebibliography}


\end{document}
